\documentclass[11pt,mathserif]{beamer}

% !!! pour compiler (en francais)
% (1) latex soutenance
% (2) dvips -Ppdf -G0 soutenance
% (3) ps2pdf soutenance.ps

% !!! pour compiler (en anglais)
% (1) latex siopt
% (2) dvips siopt.dvi -u +psfonts.cmz -o
% (3) ps2pdf siopt.ps


%% ====== packages ==
%\usepackage{../../../inputs/style-article/police-speciale-bouquin}
%\usepackage{../../../inputs/style-article/enumeration-bouquin}
%\usepackage{commande-hk-bouquin}
%\usepackage{../../../inputs/style-article/mathematique-bouquin}
%\usepackage{ifthen} % pour commande-hk-bouquin
\usepackage{amsmath,amssymb,amsfonts}
%\usepackage{style-rticle/mise-en-page-bouquin}  % un petit souci de caption
\usepackage{graphicx}
\newcommand{\tiret}{\rule[0.6ex]{1.3ex}{0.22ex}}

% !! pour le francais
\usepackage[latin1]{inputenc}
%\usepackage[ec]{aeguill}
% ou mieux \usepackage[cyr]{aeguill} si disponible
% aeguill charge ae automatiquement
% et ae charge [T1]{fontenc} automatiquement
\usepackage[french]{babel}
\usepackage{algorithmicx}
\usepackage{algpseudocode}
\usepackage{fancyvrb}
\usepackage{relsize}
\usepackage{color}



%\usepackage[T1]{fontenc}

%% ====== more def ==

\newcommand{\fleche}{\alert{$\pmb{\longrightarrow}$}~~}
\newcommand{\calM}{{\mathcal{M}}}
\newcommand{\calU}{{\mathcal{U}}}
\newcommand{\Cau}{\mathcal{C}_{n+1}}
\newcommand{\R}{\mathbb{R}}
\newcommand{\Sym}[1]{\mathcal{S}_{#1}(\R)}
\newcommand{\tran}{^{\top}}
\newcommand{\pssg}{\langle \langle}
\newcommand{\pssd}{\rangle \rangle}
\newcommand{\aronde}{\mathcal{A}}
\newcommand{\Tr}{\mathtt{Tr}}
\newcommand{\Matr}[1]{\mathcal{M}_{#1}(\R)}
\newcommand{\calV}{{\mathcal{V}}}
\newcommand{\calS}{{\mathcal{S}}}
\newcommand{\corra}{{\mathrm{corr}_a}}
\newcommand{\rmD}{{\mathrm{D}}}
\newcommand{\rmN}{{\mathrm{N}}}
\newcommand{\rmP}{{\mathrm{P}}}
\newcommand{\rmT}{{\mathrm{T}}}
\def\Sy{{\cal S}_n}
\newcommand{\accol}[1]{{\left\{ \begin{array}{ll} #1 \end{array} \right.}}
\newcommand{\prods}[2]{\langle #1,#2 \rangle}
\newcommand{\Aa}{\mathcal{A}}
\newcommand{\norm}[1]{\|#1\|}
%\DeclareMathOperator{\vect}{vect}
%\newcommand{\rint}{\textrm{ri}\,}
\newcommand{\vect}{\mathrm{vect}}
\newenvironment{disarray}%
 {\everymath{\displaystyle\everymath{}}\array}%
 {\endarray}

\catcode`\�=\active
\catcode`\�=\active
\def�{\og\ignorespaces}
\def�{{\fg}}

% et la commande Pour les guillemets
\newcommand{\guill}[1]{�#1�} % attention deja dans mycv

\renewcommand{\succeq}{\succcurlyeq}
\renewcommand{\preceq}{\preccurlyeq}

%% ====== my bearmer ==
\mode<presentation> {
\usetheme{default}    % sobre
%\usetheme{Pittsburgh}  % default avec un titre a droite
%\usetheme{Singapore}   % avec une table-of-contens-sildebar

% options
\useinnertheme[shadow]{rounded}  % les numeros
}


\usefonttheme{structurebold}



\begin{document}

%****************************************************************
% Page de presentation 
%**************************************************************
\begin{frame}
\begin{center}
{\Large Initiation a la Programmation avec Python } 
\end{center}
%\begin{center}
%\includegraphics[width=0.3\linewidth]{Fig/gdb}
%\end{center}
\begin{center}
{\large Inria Bordeaux-Sud Ouest \\ F�te de la Science 2013}
\end{center}
\end{frame}

%****************************************************************
% Plan 
%**************************************************************

\begin{frame}
\frametitle{Plan}

\begin{enumerate}
\item Qu'est ce que la programmation 
\item Exemples 
\item Python
   \begin{itemize}
    \item donn�es python
    \item structures de contr�le. 
   \end{itemize}
\end{enumerate}

\end{frame}

%%%%%%%%%%%%%%%%%%%%%%%%%%%%%%%%%%%%%%%%%%%%%%%%%%%%%%%%%%%%%%%%%%
%
%%%%%%%%%%%%%%%%%%%%%%%%%%%%%%%%%%%%%%%%%%%%%%%%%%%%%%%%%%%%%%%%%%%

%****************************************************************
%  example 1
%****************************************************************
\begin{frame}
\frametitle{Exemple 1 : somme des $n$ premier entiers}
\begin{itemize}[<+->]
  \item l'algorithme peut ressembler a ca : 
   \begin{description}
      \item[1.] dans une case m�moire (variable) je mets 1
      \item[2.] au contenu de la case m�moire j'atoute 2
      \item[2.] au contenu de la case j'atoute 3 
      \item[...] .....
      \item[n.] au contenu de la case m�moire j'atoute $n$ 
      \item[n+1.] je retourne le contenu de la case m�moire
   \end{description}
\end{itemize}
\end{frame}

%****************************************************************
%  example 1
%****************************************************************
\begin{frame}
\frametitle{Exemple 1 : somme des $n$ premier entiers}
\begin{itemize}[<+->]
  \item On peut formaliser {\bf l'algorithme} avec du {\em pseudo-code}
   \begin{algorithmic}[H]
     \State{$s \gets 1$}
     \For{$i = 2,..,n$} 
          \State{ $s \gets s + i$}
     \EndFor
     \Return{s}
   \end{algorithmic}
\end{itemize}
\end{frame}

\end{document}
