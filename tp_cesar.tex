\documentclass[11pt]{letter}

%%%%%%%%%%%%%%%%%% PACKAGES %%%%%%%%%%%%%%%%%%%%%%%%%%
\usepackage[french]{babel}
\usepackage{amssymb}
\usepackage{amsmath}
\usepackage{stmaryrd}
\usepackage{amscd}
\usepackage{enumerate}
\usepackage{graphicx} %\usepackage{epsfig}
\usepackage{amsfonts}
\usepackage[latin1]{inputenc}  
\usepackage{tikz}
\usepackage{multirow}
\usepackage{array}

%%%%%%%%%%%%%%%%%%%%%%%%%%%%%%%%%%%%%%%%%%%%%%%%%%

\begin{document}
%\pagestyle{empty}

%%%%%%%%%%%%%%%%% TITRE %%%%%%%%%%%%%%%%%%%%%%%%%%
\begin{center}
{\large Exercice sur les codes César 
\end{center}

%%%%%%%%%%%%%%%%%%%%%%%%%%%%%%%%%%%%%%%%%%%%%%%%%%%%

\begin{enumerate}
   \item écrire une fonction cesar qui prend une clef (un nombre entre 0 et 26) et un mot (une chaîne de caractères, par exemple mot='ici')
   et qui renvoie la chaine codée. Pour cela, on pourra suivre les étapes suivantes 
    \begin{itemize}
      \item crée une liste (\texttt{alpha}) qui contient l'alphabet. regarder la documentation de string.ascii\_lowercase
      \item crée la liste décalée (\texttt{decale}) de la clef (utilise une boucle, ou l'indicage des listes plus la contatenation 
      \begin{lstlistings}
        l = [1,2,3, 4]; g = l[2:] # g = [3,4]
        a = [12, 'g', -1] + ['e', '7'] #  a = [12, 'g', -1, 'e', '7']
       \end{lstlistings}
      \item crée un dictionnaire (par une boucle dic = \{\} avec pour clefs les éléments de \texttt{alpha} et pour valeurs les
      elements de \texttt{decale}
      \item avec une boucle sur la longueur de mot , crypter le mot
    \end{itemize}
\end{enumerate}





\end{document}


